\documentclass[a4paper]{article}

\usepackage{amsmath, amsthm, amssymb}
\usepackage[utf8]{inputenc}
\usepackage[T1]{fontenc}
\usepackage{color}
\usepackage{bussproofs}
\usepackage[pdftex]{graphicx}
\usepackage{fullpage}
\usepackage{cmll}

\newcommand{\interp}[1]{[\![ #1 ]\!]}
\newcommand{\wproof}[1]{{\color{blue}\mathbb{W}}[#1]}
\newcommand{\cproof}[1]{{\color{red}\mathbb{C}}[#1]}
\newcommand{\inl}[1]{\mathtt{inl}\,#1}
\newcommand{\inr}[1]{\mathtt{inr}\,#1}
\newcommand{\prf}[3]{#1 \vdash #2 : #3}
\newcommand{\tlam}[3]{\lambda #1 : #2.\, #3}
\newcommand{\letin}[3]{\mathtt{let}\ #1\ := #2\ \mathtt{in}\ #3}
\newcommand{\ifthen}[3]{\mathtt{if}\ #1\ \mathtt{then}\ #2\ \mathtt{else}\ #3}
\newcommand{\process}[2]{\langle #1 \mid #2\rangle}

\newtheorem{theorem}{Theorem}
\newtheorem{proposition}{Proposition}

\begin{document}

\section{Original Dialectica}

\subsection{Definition}

Dialectica interpretation is defined by induction on the type of the formula.
First, we fix a given type $R$, with an inhabitant $\maltese$. To any linear
type $A$ we associate two intuitionistic types $\wproof{A}$ and $\cproof{A}$.

\begin{center}
\renewcommand{\arraystretch}{1.5}
\begin{tabular}{c|c|c}
& $\wproof{\cdot}$ & $\cproof{\cdot}$ \\\hline

$1$ &
$1$ &
$R$ \\

$A \otimes B$ &
$\wproof{A} \times \wproof{B}$ &
$(\wproof{A} \Rightarrow \cproof{B}) \times (\wproof{B} \Rightarrow \cproof{A})$
\\

$A \oplus B$ &
$\wproof{A} + \wproof{B}$ &
$\cproof{A} \times \cproof{B}$ \\

$\oc{A}$ &
$\wproof{A}$ &
$\wproof{A} \Rightarrow \cproof{A}$ \\

$A^\perp$ &
$\cproof{A}$ &
$\wproof{A}$ \\

\end{tabular}
\end{center}

We also define an orthogonality on pseudoproofs: ${\perp_A} \subseteq \wproof{A}
\times \cproof{A}$ by induction on $A$.

\begin{center}
\renewcommand{\arraystretch}{3}
\begin{tabular}{ccc}

\AxiomC{\strut$\neg (x \perp_A u)$}
\UnaryInfC{\strut $u \perp_{A^\perp} x$}
\DisplayProof &

\AxiomC{\strut$\chi:R$}
\UnaryInfC{\strut $() \perp_1 \chi$}
\DisplayProof &

\AxiomC{\strut$u \perp_A \psi v$}
\AxiomC{\strut$v \perp_B \varphi u$}
\BinaryInfC{\strut $(u, v) \perp_{A \otimes B} (\varphi, \psi)$}
\DisplayProof \\

\AxiomC{\strut$u \perp_A z u$}
\UnaryInfC{\strut $u \perp_{\oc{A}} z$}
\DisplayProof &

\AxiomC{\strut$u \perp_A x$}
\UnaryInfC{\strut $\inl u \perp_{A \oplus B} (x, y)$}
\DisplayProof &

\AxiomC{\strut$v \perp_A y$}
\UnaryInfC{\strut $\inr v \perp_{A \oplus B} (x, y)$}
\DisplayProof
\end{tabular}
\end{center}

\begin{theorem}
The orthogonality relation is decidable.
\end{theorem}


A term $u : \wproof{A}$ is said to be valid, noted $u \Vdash A$, whenever
$$ \forall x : \cproof{A}.\, u \perp_A x$$

For any $A$, $\wproof{A}$ is inhabited by a canonical term $\maltese_{A}$:

\begin{center}
\renewcommand{\arraystretch}{1.5}
\begin{tabular}{c|c|c}
& $\maltese_{-}$
& $\maltese_{-^\perp}$ \\\hline

$1$ & $()$ & $\maltese$ \\

$A \otimes B$ &
$ (\maltese_A, \maltese_B)$ &
$\left\{
  \begin{array}{c}
  \lambda u : \wproof{A}.\,\maltese_{B^\perp} \\
  \lambda v : \wproof{B}.\,\maltese_{A^\perp}
  \end{array}
\right\}$ \\

$A \oplus B$ &
$ \inl{(\maltese_A)}$ &
$(\maltese_{A^\perp}, \maltese_{B^\perp})$\\

$\oc{A}$ &
$ \maltese_A$ &
$\lambda u : \wproof{A}.\,\maltese_{A^\perp}$

\end{tabular}
\end{center}

\subsection{Translating CBN $\lambda$-calculus}

We note $t^+$ (resp. $t^-$) for the first (resp. second) component of the tensor
counterproof (which is also the same component of the arrow proof).

This translation is based on the CBN interpretation of intuitionistic logic into
linear logic:
$$\interp{\Gamma \vdash A}_n =
  \oc{\interp{\Gamma}}_n \vdash_{\mathrm{LL}} \interp{A}_n$$
$$\interp{A \Rightarrow B}_n = \oc{\interp{A}}_n \multimap \interp{A}_n$$

To every simply-typed $\lambda$-term, we associate its Dialectica translation by
induction on its typing derivation as follows:

$$
\renewcommand{\arraystretch}{1.5}
\begin{array}{rcl}

\interp{\prf{\oc\Gamma, x: \oc{A}}{x}{A}} & = &
\left\{\begin{array}{l}
  \tlam{(\gamma, u)}{\wproof{\oc\Gamma} \times \wproof{!A}}{u}\\
  \tlam{x}{\cproof{!A}}{
    \left\{\begin{array}{l}
      \tlam{\gamma}{\wproof{\oc{\Gamma}}}{\tlam{u}{\wproof{A}}{x}} \\
      \tlam{u}{\wproof{\oc{A}}}{\maltese_{\oc\Gamma}}
    \end{array}\right\}

  }
\end{array}\right\} \\

\interp{\prf{\oc\Gamma, y: \oc{B}}{x}{A}} & = &
\left\{\begin{array}{l}
  \tlam{(\gamma, v)}
    {\wproof{\oc\Gamma} \times \wproof{!B}}
      {\interp{\prf{\oc\Gamma}{x}{A}}^+(\gamma)}\\
  \tlam{x}{\cproof{!A}}{
    \left\{\begin{array}{l}
      \tlam{\gamma}{\wproof{\oc{\Gamma}}}{\maltese_{\oc{B}}}\\
      \tlam{v}{\wproof{\oc{B}}}
        {\interp{\prf{\oc\Gamma}{x}{A}}^- (x)}
    \end{array}\right\}

  }
\end{array}\right\} \\

\interp{\prf{\oc\Gamma}{\tlam{x}{\oc{A}}{t}}{\oc{A} \multimap B}} & = &
\left\{\begin{array}{l}
  \tlam{\gamma}{\wproof{\oc{\Gamma}}}{
    \left\{\begin{array}{l}
      \tlam{u}{\wproof{\oc{A}}}{\interp{\prf{\oc\Gamma, x : A}{t}{B}}^+(\gamma,
u)}\\
      \tlam{y}{\cproof{B}}
        {\left({\interp{\prf{\oc\Gamma, x : A}{t}{B}}^-(y)}\right)^+(\gamma)}
    \end{array}\right\}} \\
  \tlam{(u, y)}{\wproof{\oc{A}} \times \cproof{B}}
    \left({\interp{\prf{\oc\Gamma, x : A}{t}{B}}^-(y)}\right)^-(u)
\end{array}\right\} \\

\interp{\prf{\oc\Gamma}{t\, u}{B}} & = &
% \interp{\prf{\oc\Gamma}{t}{\oc{A} \multimap B}}
% \interp{\prf{\oc\Gamma}{u}{A}}
\left\{\begin{array}{l}
  \tlam{\gamma}{\wproof{\oc{\Gamma}}}{
    \left({\interp{\prf{\oc\Gamma}{t}{\oc{A} \multimap B}}^+(\gamma)}\right)^+
    \left({\interp{\prf{\oc\Gamma}{u}{A}}^+(\gamma)}\right)
  }\\
  \tlam{y}{\cproof{B}}{
    \tlam{\gamma}{\wproof{\oc\Gamma}}{\\\quad
      \letin
        {\mathbf{u} : \wproof{A}}
        {\interp{\prf{\oc\Gamma}{u}{A}}^+(\gamma)}
        {\\\quad
      \letin
        {\mathbf{t} : \wproof{\oc{A}\multimap B}}
        {\interp{\prf{\oc\Gamma}{t}{\oc{A}\multimap B}}^+(\gamma)}
        {\\\quad
      \letin
        {\kappa}
        {{\interp{\prf{\oc\Gamma}{u}{A}}}^-(\mathbf{t}^-\ y\ \mathbf{u})}
        {\\\quad
          \ifthen{\gamma\perp_{\oc{\Gamma}}\kappa}{\kappa\ \gamma}
          {\\\qquad
            {\interp{\prf{\oc\Gamma}{t}{\oc{A}\multimap B}}^-(\mathbf{u}, y)\ 
              \gamma}}
        }
        }
      }
    }
  }
\end{array}\right\}

\end{array}$$

\begin{theorem}[Soundness]
For any $\prf{!\Gamma}{t}{A}$, $\interp{\prf{!\Gamma}{t}{A}} \Vdash \oc{\Gamma} 
\multimap A$.
\end{theorem}

\subsection{Rephrasing using nice isomorphisms}

Let $\Gamma \equiv \Gamma_1, \hdots \Gamma_n$. If we stick to CBN
$\lambda$-calculus, we can rephrase the types of the morphisms as follows:

$$
\begin{array}{rcl}

\wproof{\oc\Gamma\multimap A} & \equiv &
(\wproof{\Gamma} \Rightarrow \wproof{A}) \times (\cproof{A} \Rightarrow 
\cproof{\oc \Gamma}) \\

& \cong &
(\wproof{\Gamma} \Rightarrow
\wproof{A}) \times (\cproof{A} \Rightarrow \Pi_i 
(\wproof{\Gamma} \Rightarrow \cproof{\Gamma_i})) \\

& \cong &
(\wproof{\Gamma} \Rightarrow
\wproof{A}) \times (\Pi_i \wproof{\Gamma} \Rightarrow \cproof{A} \Rightarrow
\cproof{\Gamma_i}) \\

& \cong &
\wproof{\Gamma} \Rightarrow \wproof{A} \times \Pi_i(\cproof{A} \Rightarrow 
\cproof{\Gamma_i}) \\

\end{array}
$$

That is, a translated morphism is isomorphic to a function waiting for its 
closure arguments and returning the result of the computation $t^\bullet$, 
together with, for every free variable $x : \Gamma_i$, a stack transformer $t_x 
: \cproof{A} \Rightarrow \cproof{\Gamma_i}$.

What does this transformer do? Actually, it is quite simple: $t_x(\pi)$ 
returns the first valid stack encountered by the variable $x$ while 
evaluating $\langle t \mid \pi \rangle$, as explained in the following section. 
For now, we explicit the isomorphic translation of CBN $\lambda$-calculus.

\begin{proposition}
We have $t \Vdash \oc{\Gamma} \multimap A$ whenever:
$$\forall \vec{v} : \wproof{\vec{\Gamma}}.\, \forall \pi : \cproof{A}.\,
\left(\bigwedge_i {v_i\perp_{\Gamma_i} t_{x_i}\, \vec{v}\, \pi}\right) 
\Rightarrow t^\bullet \vec{v} \perp_A \pi
$$
\end{proposition}

\begin{proposition}
We explicit the previous translation using the isomorphisms. For the sake of 
clarity, it works on open terms. Let $\Gamma \vdash t : A$ an open term with
unbound variables $\{ x_i : \Gamma_i \mid i \le n \}$.

\begin{itemize}
  \item The $(-)^\bullet$ transformation associates to $t$ an open term 
  $\vec{x} : \wproof{\Gamma} \vdash t^\bullet : \wproof{A}$.
  \item The $(-)_{x_i}$ transformation associates to $t$ an open term $\vec{x} 
:   \wproof{\Gamma} \vdash t_{x_i} : \cproof{A} \Rightarrow \cproof{\Gamma_i}$.
\end{itemize}


$$
\renewcommand{\arraystretch}{1.5}
\begin{array}{lcl}
\interp{\Gamma, x : A\vdash x : A}[\vec{x}, x] &=&
  \left\{\begin{array}{lcl}
    x^\bullet &:=& x \\
    &:& \wproof{A}\\
    x_{x_i} &:=& \lambda \pi.\, \maltese\\
    &:& \cproof{A} \Rightarrow \cproof{\Gamma_i}\\
    x_{x} &:=& \lambda \pi.\, \pi\\
    &:& \cproof{A} \Rightarrow \cproof{A}\\
  \end{array}\right\}\\

\interp{\Gamma \vdash \tlam{x}{A}{t} : A \Rightarrow B}[\vec{x}] &=&
  \left\{\begin{array}{lcl}
    (\lambda x.\,t)^{\bullet+} &:=& \lambda x.\, t^\bullet \\
    &:& \wproof{A} \Rightarrow \wproof{B}\\
    (\lambda x.\,t)^{\bullet-} &:=& \lambda \pi x.\, t_x\,\pi\\
    &:& \cproof{B} \Rightarrow \wproof{A} \Rightarrow \cproof{A}\\
    (\lambda x.\,t)_{x_i} &:=& \lambda (x, \pi).\, t_{x_i}\, \pi \\
    &:& \wproof{A}\otimes\cproof{B} \Rightarrow \cproof{\Gamma_i}\\
  \end{array}\right\}\\

\interp{\Gamma \vdash t\, u : B}[\vec{x}] &=&
  \left\{\begin{array}{lcl}
    (t\, u)^{\bullet} &:=& t^{\bullet+}\, u^\bullet\\
    &:& \wproof{B}\\
    (t\, u)_{x_i} &:=& \lambda \pi.[t_{x_i}\, (u^\bullet, \pi)] @ [u_{x_i} 
    (t^{\bullet-} \pi)] \\
    &:& \cproof{B} \Rightarrow \cproof{\Gamma_i}\\
  \end{array}\right\}\\

\end{array}$$

\noindent where $\pi @ \rho = \left\{
  \begin{array}{ll}
    \rho & \text{if $\pi = \maltese$} \\
    \pi & \text{otherwise}
  \end{array}\right.$

\end{proposition}

The important thing here is to have the following property: if $t \perp_A \pi @
\rho$, then $t \perp_A \pi$ and $t \perp_A \rho$.

\subsection{The abstract machine}

This machine will be a plain Krivine abstract machine using explicit closures 
instead of direct substitutions.

$$\begin{array}{lccl}
\text{Terms} & t &::=& x \mid \lambda x.t \mid t\, u\\
\text{Stacks} & \pi &::=& \varepsilon \mid c \cdot \pi \\
\text{Closures} & c &::=& (t, \sigma) \\
\text{Environments} & \sigma &::=& \varnothing \mid \sigma + (x := c)
\end{array}$$\bigskip

The machine evaluates processes $\process{(t, \sigma)}{\pi}$ with the
following reduction rules:

$$\begin{array}{rcl}
\process{(x, \sigma + (x := c))}{\pi} & \rightarrow &
  \process{c}{\pi}\\
\process{(x, \sigma + (y := c))}{\pi} & \rightarrow &
  \process{(x, \sigma)}{\pi}\\
\process{(\lambda x.t, \sigma)}{c \cdot \pi} & \rightarrow &
  \process{(t, \sigma + (x := c)}{\pi}\\
\process{(t\, u, \sigma)}{\pi} & \rightarrow &
  \process{(t, \sigma)}{(u, \sigma) \cdot \pi}\\
&\hdots&
\end{array}$$

$$\begin{array}{lccl}
\text{Terms} & t &::=&
  \hdots \mid \mathtt{set}\ x\ \mathtt{in}\ t \\
\text{Closures} & c &::=& \hdots \mid \maltese \\
\end{array}$$

\begin{center}
\AxiomC{\strut$\process{(t,\sigma + (x := \maltese))}{\pi} \rightarrow^* 
\process{(x, \sigma + (x := \maltese)}{\rho}$}
\UnaryInfC{\strut$\process{(\mathtt{set}\ x\ \mathtt{in}\ t, \sigma)}{c \cdot 
  \pi} \rightarrow \process{c}{\rho \odot \pi}$}
\DisplayProof
\end{center}

\noindent where $\odot$ denotes the list concatenation, i.e.
$$\epsilon \odot \pi = \pi$$
$$(c \cdot \rho)\odot \pi = c \cdot (\rho \odot \pi)$$

\subsection{Translating CBV $\lambda$-calculus}

We use this time the boring translation. We make the difference between values 
$\Gamma \vdash_\mathbb{V} v : A$ and computations $\Gamma \vdash t : A$ 
explicit, as well as the injection from values to computations.

$$\interp{\Gamma\vdash_{\mathbb{V}} A}_v = 
\oc\interp{\Gamma}_v\vdash\interp{A}_v$$
$$\interp{\Gamma \vdash A}_v = \oc\interp{\Gamma}_v \vdash \oc\interp{A}_v$$
$$\interp{A \Rightarrow B}_v = \oc{\interp{A}_v} \multimap \oc{\interp{B}_v}$$

Rewriting using the same isomorphisms as the call-by-name case, we obtain:
$$\begin{array}{lcl}
  \interp{\Gamma\vdash A}_v
  &=& {\oc\interp{\Gamma}_v} \multimap \oc{\interp{A}_v}\\

  &\cong& (\wproof{\oc\interp{\Gamma}_v} \Rightarrow \wproof{\oc\interp{A}_v})
  \times (\cproof{\oc\interp{A}_v} \Rightarrow \cproof{\oc\interp{\Gamma}_v})\\

  &\cong& (\wproof{\interp{\Gamma}_v} \Rightarrow \wproof{\interp{A}_v})
  \times ((\wproof{\interp{A}_v} \Rightarrow \cproof{\interp{A}_v}) \Rightarrow
  (\Pi_i (\wproof{\interp{\Gamma}_v}
    \Rightarrow \cproof{\interp{\Gamma_i}_v}))\\

  &\cong& \wproof{\interp{\Gamma}_v} \Rightarrow
    \wproof{\interp{A}_v} \times \Pi_i
    ((\wproof{\interp{A}_v} \Rightarrow \cproof{\interp{A}_v})
      \Rightarrow \cproof{\interp{\Gamma_i}_v})
\end{array}$$

\newcommand{\kont}[1]{\mathfrak{K}[#1]}

Now, instead of a stack transformer, each free variable $x_i : \Gamma_i$ comes
with a continuation transformer $t_{x_i}$ which takes the current continuation 
of type $(\wproof{\interp{A}_v} \Rightarrow \cproof{\interp{A}_v})$ and must 
output a stack for $\Gamma_i$. Let us note $\kont{A}$ the type of such
continuations, so that we have:

$$
  \interp{\Gamma\vdash A}_v \cong \wproof{\interp{\Gamma}_v} \Rightarrow
    \wproof{\interp{A}_v} \times \Pi_i
    (\kont{A} \Rightarrow \cproof{\interp{\Gamma_i}_v})\\
$$

Validity is adapted as $t \Vdash \Gamma \vdash A$ whenever for any $\vec{u} : 
\wproof{\Gamma}$ and $k : \kont{A}$:
$$
  \left(\bigwedge_i u_i \perp_{\Gamma_i} t_{x_i}[\vec{x} := \vec{u}]\, k\right)
    \Rightarrow t^\bullet[\vec{x} := \vec{u}] \perp_A k\,
      (t^\bullet [\vec{x} := \vec{u}]) $$

Likewise we have $v \Vdash \Gamma \vdash_{\mathbb{V}} A$ whenever for any 
$\vec{u} : \wproof{\Gamma}$ and $\pi : \cproof{A}$:

$$
  \left(\bigwedge_i u_i \perp_{\Gamma_i} v_{x_i}[\vec{x} := \vec{u}]\,\pi\right)
    \Rightarrow v^\bullet[\vec{x} := \vec{u}] \perp_A \pi $$

And the transformation is:

$$
\renewcommand{\arraystretch}{1.5}
\begin{array}{lcl}
\interp{\Gamma, x : A\vdash_{\mathbb{V}} x : A}[\vec{x}, x] &=&
  \left\{\begin{array}{lcl}
    x^\bullet &:=& x \\
    &:& \wproof{A}\\
    x_{x_i} &:=& \lambda k.\, \maltese\\
    &:& \cproof{A} \Rightarrow \cproof{\Gamma_i}\\
    x_{x} &:=& \lambda \pi.\, \pi \\
    &:& \cproof{A} \Rightarrow \cproof{A}\\
  \end{array}\right\}\\

\interp{\Gamma \vdash_{\mathbb{V}} \tlam{x}{A}{t} : A \Rightarrow B}[\vec{x}]&=&
  \left\{\begin{array}{lcl}
    (\lambda x.\,t)^{\bullet+} &:=& \lambda x.\, t^\bullet \\
    &:& \wproof{A} \Rightarrow \wproof{B}\\
    (\lambda x.\,t)^{\bullet-} &:=& \lambda kx.\, t_x\,k\\
    &:& \mathfrak{K}[B]\Rightarrow\mathfrak{K}[A]\\
    (\lambda x.\,t)_{x_i} &:=& \lambda (x, k).\, t_{x_i}\, k\\
    &:& \wproof{A}\otimes \kont{B}
      \Rightarrow \cproof{\Gamma_i}\\
  \end{array}\right\}\\
\end{array}$$

$$
\renewcommand{\arraystretch}{1.5}
\begin{array}{lcl}
\interp{\Gamma \vdash v : A}[\vec{x}] &=&
  \left\{\begin{array}{lcl}
    {[v]}^\bullet &:=& v^\bullet \\
    &:& \wproof{A}\\
    {[v]}_{x} &:=& \lambda k.\, v_x\,(k\, v^\bullet) \\
    &:& \kont{A} \Rightarrow \cproof{\Gamma_i}\\
  \end{array}\right\}\\

\interp{\Gamma \vdash t\, u : B}[\vec{x}] &=&
  \left\{\begin{array}{lcl}
    (t\, u)^{\bullet} &:=& t^{\bullet+}\, u^\bullet\\
    &:& \wproof{B}\\
    (t\, u)_{x_i} &:=& \lambda k.\, u_{x_i}(t^{\bullet-}\, k) @
      (t_{x_i}\, (\lambda f.\,(u^\bullet, k)))\\
    &:& \kont{B} \Rightarrow \cproof{\Gamma_i}\\
  \end{array}\right\}\\

\end{array}$$

\begin{theorem}[Soundness]

If $\Gamma \vdash t : A$ then $t \Vdash \Gamma \vdash A$.

\end{theorem}

If we focus a bit on the translation of the arrow, we may remark that:
$$\interp{A\Rightarrow B}_v \cong
  \interp{A}_v \Rightarrow M_{A, B} \interp{B}_v$$
\noindent where
$$M_{A, B}(X) = X \otimes (X \Rightarrow \cproof{B}) \Rightarrow \cproof{A}$$

\section{The ListT variant}

Actually, we can tweak the interpretation of $\oc A$ types to be a little more
algebraic-friendly. This gives out a better model. For this to work, we first 
need to define an inductive type:

$$\begin{array}{rcl}
\mathtt{rlist}\ I\ A & ::= &
  \mathtt{Node} : I \Rightarrow \mathtt{rnode}\ I\ A\\
\mathtt{rnode}\ I\ A & ::= &
    \mathtt{Nil} \mid \mathtt{Cons} : A \times \mathtt{rlist}\ I\ A
\end{array}
$$

This type is formally the result of combining the list monad transformer with
the reader monad on type $I$, i.e. $\mathtt{rlist}\ I\ A \cong \mathtt{listT}\ 
(\mathtt{reader}\ I)\ A$. We now change the counteproofs of bang types as:

$$\cproof{\oc A} \equiv \mathtt{rlist}\ \wproof{A}\ \cproof{A}$$

Orthogonality is also adapted:

\begin{center}
\renewcommand{\arraystretch}{3}
\begin{tabular}{cc}

\AxiomC{\strut$\mu\, u \rightarrow^* \mathtt{Nil}$}
\UnaryInfC{\strut $u \perp_{\oc A} \mathtt{Node}\ \mu$}
\DisplayProof &

\AxiomC{\strut$\mu\, u \rightarrow^* \mathtt{Cons}\ (x, l)$}
\AxiomC{\strut$u \perp_{A} x$}
\AxiomC{\strut$u \perp_{\oc A} l$}
\TrinaryInfC{\strut $u \perp_{\oc A} \mathtt{Node}\ \mu$}
\DisplayProof
\end{tabular}
\end{center}

This choice makes the $\oc$ monad better behaved. In particular, we explicitely 
get the back-and-forth mechanism used when accessing variables. Each time a
variable has to be accessed, the $\mathtt{rlist}\ \wproof{A}\ \cproof{A}$ asks
for a content and returns the current stack (the $\cproof{A}$ part) together
with a continuation to the remaining of the computation.

\end{document}
