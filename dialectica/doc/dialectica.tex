\documentclass[a4paper]{article}

\usepackage{amsmath, amsthm, amssymb}
\usepackage[utf8]{inputenc}
\usepackage[T1]{fontenc}
\usepackage{color}
\usepackage{bussproofs}
\usepackage[pdftex]{graphicx}
\usepackage{fullpage}
\usepackage{cmll}

\newcommand{\interp}[1]{[\![ #1 ]\!]}
\newcommand{\wproof}[1]{{\color{blue}\mathbb{W}}[#1]}
\newcommand{\cproof}[1]{{\color{red}\mathbb{C}}[#1]}
\newcommand{\inl}[1]{\mathtt{inl}\,#1}
\newcommand{\inr}[1]{\mathtt{inr}\,#1}
\newcommand{\prf}[3]{#1 \vdash #2 : #3}
\newcommand{\tlam}[3]{\lambda #1 : #2.\, #3}
\newcommand{\letin}[3]{\mathtt{let}\ #1\ := #2\ \mathtt{in}\ #3}
\newcommand{\ifthen}[3]{\mathtt{if}\ #1\ \mathtt{then}\ #2\ \mathtt{else}\ #3}
\newcommand{\process}[2]{\langle #1 \mid #2\rangle}

\newtheorem{theorem}{Theorem}

\begin{document}

\section{Original Dialectica}

\subsection{Definition}

Dialectica interpretation is defined by induction on the type of the formula.
First, we fix a given type $R$, with an inhabitant $\maltese$. To any linear
type $A$ we associate two intuitionistic types $\wproof{A}$ and $\cproof{A}$.

\begin{center}
\renewcommand{\arraystretch}{1.5}
\begin{tabular}{c|c|c}
& $\wproof{\cdot}$ & $\cproof{\cdot}$ \\\hline

$1$ &
$1$ &
$R$ \\

$A \otimes B$ &
$\wproof{A} \times \wproof{B}$ &
$(\wproof{A} \Rightarrow \cproof{B}) \times (\wproof{B} \Rightarrow \cproof{A})$
\\

$A \oplus B$ &
$\wproof{A} + \wproof{B}$ &
$\cproof{A} \times \cproof{B}$ \\

$\oc{A}$ &
$\wproof{A}$ &
$\wproof{A} \Rightarrow \cproof{A}$ \\

$A^\perp$ &
$\cproof{A}$ &
$\wproof{A}$ \\

\end{tabular}
\end{center}

We also define an orthogonality on pseudoproofs: ${\perp_A} \subseteq \wproof{A}
\times \cproof{A}$ by induction on $A$.

\begin{center}
\renewcommand{\arraystretch}{3}
\begin{tabular}{ccc}

\AxiomC{\strut$\neg (x \perp_A u)$}
\UnaryInfC{\strut $u \perp_{A^\perp} x$}
\DisplayProof &

\AxiomC{\strut$\chi:R$}
\UnaryInfC{\strut $() \perp_1 \chi$}
\DisplayProof &

\AxiomC{\strut$u \perp_A \psi v$}
\AxiomC{\strut$v \perp_B \varphi u$}
\BinaryInfC{\strut $(u, v) \perp_{A \otimes B} (\varphi, \psi)$}
\DisplayProof \\

\AxiomC{\strut$u \perp_A z u$}
\UnaryInfC{\strut $u \perp_{\oc{A}} z$}
\DisplayProof &

\AxiomC{\strut$u \perp_A x$}
\UnaryInfC{\strut $\inl u \perp_{A \oplus B} (x, y)$}
\DisplayProof &

\AxiomC{\strut$v \perp_A y$}
\UnaryInfC{\strut $\inr v \perp_{A \oplus B} (x, y)$}
\DisplayProof
\end{tabular}
\end{center}

\begin{theorem}
The orthogonality relation is decidable.
\end{theorem}


A term $u : \wproof{A}$ is said to be valid, noted $u \Vdash A$, whenever
$$ \forall x : \cproof{A}.\, u \perp_A x$$

For any $A$, $\wproof{A}$ is inhabited by a canonical term $\maltese_{A}$:

\begin{center}
\renewcommand{\arraystretch}{1.5}
\begin{tabular}{c|c|c}
& $\maltese_{-}$
& $\maltese_{-^\perp}$ \\\hline

$1$ & $()$ & $\maltese$ \\

$A \otimes B$ &
$ (\maltese_A, \maltese_B)$ &
$\left\{
  \begin{array}{c}
  \lambda u : \wproof{A}.\,\maltese_{B^\perp} \\
  \lambda v : \wproof{B}.\,\maltese_{A^\perp}
  \end{array}
\right\}$ \\

$A \oplus B$ &
$ \inl{(\maltese_A)}$ &
$(\maltese_{A^\perp}, \maltese_{B^\perp})$\\

$\oc{A}$ &
$ \maltese_A$ &
$\lambda u : \wproof{A}.\,\maltese_{A^\perp}$

\end{tabular}
\end{center}

\subsection{Translating CBN $\lambda$-calculus}

We note $t^+$ (resp. $t^-$) for the first (resp. second) component of the tensor
counterproof (which is also the same component of the arrow proof).

To every simply-typed $\lambda$-term, we associate its Dialectica translation by
induction on its typing derivation as follows:

$$
\renewcommand{\arraystretch}{1.5}
\begin{array}{rcl}

\interp{\prf{\oc\Gamma, x: \oc{A}}{x}{A}} & = &
\left\{\begin{array}{l}
  \tlam{(\gamma, u)}{\wproof{\oc\Gamma} \times \wproof{!A}}{u}\\
  \tlam{x}{\cproof{!A}}{
    \left\{\begin{array}{l}
      \tlam{\gamma}{\wproof{\oc{\Gamma}}}{\tlam{u}{\wproof{A}}{x}} \\
      \tlam{u}{\wproof{\oc{A}}}{\maltese_{\oc\Gamma}}
    \end{array}\right\}

  }
\end{array}\right\} \\

\interp{\prf{\oc\Gamma, y: \oc{B}}{x}{A}} & = &
\left\{\begin{array}{l}
  \tlam{(\gamma, v)}
    {\wproof{\oc\Gamma} \times \wproof{!B}}
      {\interp{\prf{\oc\Gamma}{x}{A}}^+(\gamma)}\\
  \tlam{x}{\cproof{!A}}{
    \left\{\begin{array}{l}
      \tlam{\gamma}{\wproof{\oc{\Gamma}}}{\maltese_{\oc{B}}}\\
      \tlam{v}{\wproof{\oc{B}}}
        {\interp{\prf{\oc\Gamma}{x}{A}}^- (x)}
    \end{array}\right\}

  }
\end{array}\right\} \\

\interp{\prf{\oc\Gamma}{\tlam{x}{\oc{A}}{t}}{\oc{A} \multimap B}} & = &
\left\{\begin{array}{l}
  \tlam{\gamma}{\wproof{\oc{\Gamma}}}{
    \left\{\begin{array}{l}
      \tlam{u}{\wproof{\oc{A}}}{\interp{\prf{\oc\Gamma, x : A}{t}{B}}^+(\gamma,
u)}\\
      \tlam{y}{\cproof{B}}
        {\left({\interp{\prf{\oc\Gamma, x : A}{t}{B}}^-(y)}\right)^+(\gamma)}
    \end{array}\right\}} \\
  \tlam{(u, y)}{\wproof{\oc{A}} \times \cproof{B}}
    \left({\interp{\prf{\oc\Gamma, x : A}{t}{B}}^-(y)}\right)^-(u)
\end{array}\right\} \\

\interp{\prf{\oc\Gamma}{t\, u}{B}} & = &
% \interp{\prf{\oc\Gamma}{t}{\oc{A} \multimap B}}
% \interp{\prf{\oc\Gamma}{u}{A}}
\left\{\begin{array}{l}
  \tlam{\gamma}{\wproof{\oc{\Gamma}}}{
    \left({\interp{\prf{\oc\Gamma}{t}{\oc{A} \multimap B}}^+(\gamma)}\right)^+
    \left({\interp{\prf{\oc\Gamma}{u}{A}}^+(\gamma)}\right)
  }\\
  \tlam{y}{\cproof{B}}{
    \tlam{\gamma}{\wproof{\oc\Gamma}}{\\\quad
      \letin
        {\mathbf{u} : \wproof{A}}
        {\interp{\prf{\oc\Gamma}{u}{A}}^+(\gamma)}
        {\\\quad
      \letin
        {\mathbf{t} : \wproof{\oc{A}\multimap B}}
        {\interp{\prf{\oc\Gamma}{t}{\oc{A}\multimap B}}^+(\gamma)}
        {\\\quad
      \letin
        {\kappa}
        {{\interp{\prf{\oc\Gamma}{u}{A}}}^-(\mathbf{t}^-\ y\ \mathbf{u})}
        {\\\quad
          \ifthen{\gamma\perp_{\oc{\Gamma}}\kappa}{\kappa\ \gamma}
          {\\\qquad
            {\interp{\prf{\oc\Gamma}{t}{\oc{A}\multimap B}}^-(\mathbf{u}, y)\ 
              \gamma}}
        }
        }
      }
    }
  }
\end{array}\right\}

\end{array}$$

\begin{theorem}[Soundness]
For any $\prf{!\Gamma}{t}{A}$, $\interp{\prf{!\Gamma}{t}{A}} \Vdash \oc{\Gamma} 
\multimap A$.
\end{theorem}

Let $\Gamma \equiv \Gamma_1, \hdots \Gamma_n$. If we stick to CBN 
$\lambda$-calculus, we can rephrase the types of the morphisms as follows:

$$
\begin{array}{rcl}

\wproof{\oc\Gamma\multimap A} & \equiv &
(\wproof{\Gamma} \Rightarrow \wproof{A}) \times (\cproof{A} \Rightarrow 
\cproof{\oc \Gamma}) \\

& \cong &
(\wproof{\Gamma} \Rightarrow
\wproof{A}) \times (\cproof{A} \Rightarrow \Pi_i 
(\wproof{\Gamma} \Rightarrow \cproof{\Gamma_i})) \\

& \cong &
\wproof{\Gamma} \Rightarrow \wproof{A} \times \Pi_i(\cproof{A} \Rightarrow 
\cproof{\Gamma_i}) \\

\end{array}
$$

That is, a translated morphism is isomorphic to a function waiting for its 
closure arguments and returning the result of the computation, together with, 
for every free variable $x : \Gamma_i$, a stack transformer $t_x : \cproof{A} 
\Rightarrow \cproof{\Gamma_i}$.

What does this transformer do? Actually, it is quite simple: $t_x(\pi)$ 
returns the first valid stack encountered by the variable $x$ while evaluating 
$\langle t \mid \pi \rangle$, as explained in the following section.

\subsection{The abstract machine}

This machine will be a plain Krivine abstract machine using explicit closures 
instead of direct substitutions.

$$\begin{array}{lccl}
\text{Terms} & t &::=& x \mid \lambda x.t \mid t\, u\\
\text{Stacks} & \pi &::=& \varepsilon \mid c \cdot \pi \\
\text{Closures} & c &::=& (t, \sigma) \\
\text{Environments} & \sigma &::=& \varnothing \mid \sigma + (x := c)
\end{array}$$\bigskip

The machine evaluates processes $\process{(t, \sigma)}{\pi}$ with the
following reduction rules:

$$\begin{array}{rcl}
\process{(\lambda x.t, \sigma)}{c \cdot \pi} & \rightarrow &
  \process{(t, \sigma + (x := c)}{\pi}\\
\process{(t\, u, \sigma)}{\pi} & \rightarrow &
  \process{(t, \sigma)}{(u, \sigma) \cdot \pi}\\
\end{array}$$

$$\begin{array}{lccl}
\text{Terms} & t &::=& \hdots \mid \mathtt{set}\ x\ \mathtt{in}\ t\\
\end{array}$$\bigskip

\section{The ListT variant}

Actually, we can tweak the interpretation of $\oc A$ types to be a little more
algebraic-friendly. This gives out a better model. For this to work, we first 
need to define an inductive type:

$$\begin{array}{rcl}
\mathtt{rlist}\ I\ A & ::= &
  \mathtt{Node} : I \Rightarrow \mathtt{rnode}\ I\ A\\
\mathtt{rnode}\ I\ A & ::= &
    \mathtt{Nil} \mid \mathtt{Cons} : A \times \mathtt{rlist}\ I\ A
\end{array}
$$

This type is formally the result of combining the list monad transformer with
the reader monad on type $I$, i.e. $\mathtt{rlist}\ I\ A \cong \mathtt{listT}\ 
(\mathtt{reader}\ I)\ A$. We now change the counteproofs of bang types as:

$$\cproof{\oc A} \equiv \mathtt{rlist}\ \wproof{A}\ \cproof{A}$$

Orthogonality is also adapted:

\begin{center}
\renewcommand{\arraystretch}{3}
\begin{tabular}{cc}

\AxiomC{\strut$\mu u \rightarrow^* \mathtt{Nil}$}
\UnaryInfC{\strut $u \perp_{\oc A} \mathtt{Node}\ \mu$}
\DisplayProof &

\AxiomC{\strut$\mu u \rightarrow^* \mathtt{Cons}\ (x, l)$}
\AxiomC{\strut$u \perp_{A} x$}
\AxiomC{\strut$u \perp_{\oc A} l$}
\TrinaryInfC{\strut $u \perp_{\oc A} \mathtt{Node}\ \mu$}
\DisplayProof
\end{tabular}
\end{center}

\end{document}
